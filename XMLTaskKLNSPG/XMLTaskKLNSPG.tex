\documentclass[12pt]{report}
\usepackage[english, magyar]{babel}
\usepackage{t1enc}
\frenchspacing

\usepackage[margin=2cm, top=5cm, bottom=2.5cm, bindingoffset=0cm]{geometry}
\usepackage{graphicx}

\usepackage{hyperref}
\hypersetup{hidelinks}

\usepackage{xcolor,listings}
\usepackage{textcomp}
\usepackage{color}
\usepackage{listingsutf8}

\definecolor{codegreen}{rgb}{0,0.6,0}
\definecolor{codegray}{rgb}{0.5,0.5,0.5}
\definecolor{codepurple}{HTML}{C42043}
\definecolor{backcolour}{HTML}{F2F2F2}
\definecolor{bookColor}{cmyk}{0,0,0,0.90}  
\definecolor{xmltagcolor}{rgb}{0,0,1}
\definecolor{xmltagcolor}{rgb}{0,0,1}
\definecolor{xmlcommentcolor}{rgb}{0,0.6,0}
\definecolor{xmlstringcolor}{rgb}{0.6,0,0}
\color{bookColor}
\lstset{upquote=true}

\lstdefinestyle{mystyle}{
	language=XML, % Setting the language as XML
	basicstyle=\ttfamily\footnotesize, % Setting the basic style
	morestring=[b]", % Strings are in double quotes
	morestring=[s][\color{xmltagcolor}]{<}{>}, % Coloring XML tags blue
	morecomment=[s][\color{xmlcommentcolor}]{<!--}{-->}, % Coloring comments
	showstringspaces=false, % Not showing spaces in strings
	breaklines=true, % Break long lines
	breakatwhitespace=true, % Break lines at whitespace
	tabsize=2, % Setting tab size
	captionpos=b, % Caption position at bottom
	extendedchars=true, % Allowing extended characters
	keepspaces=true, % Keeping spaces for formatting
}

\lstset{style=mystyle} % Applying the style globally

\usepackage{fancyhdr}
\fancypagestyle{plain}{
\fancyhf{}
\fancyhead[R]{\leftmark}
\fancyhead[L]{\thepage}
\fancyfoot[C]{Adatkezelés XML környezetben}}

\fancyhead[R]{\leftmark}
\fancyhead[L]{\thepage}
\fancyfoot[C]{Adatkezelés XML környezetben}

\begin{document}
	\pagestyle{fancy}
	\title{\Huge JEGYZŐKÖNYV \\ \LARGE Adatkezelés XML környezetben}
	\author{\Large Féléves feladat: Állatkerthálózat}
	\date{\vspace{250px}
		\begin{flushleft}
			Készítette: \textbf{Martinák Mátyás}\\
			Neptunkód: \textbf{KLNSPG}\\
			Dátum: \textbf{2023. 10. 25.}
		\end{flushleft}
		\vspace{15px}
		\begin{center}
			\textbf{Miskolc, 2023}
		\end{center}}
	\maketitle
	
\tableofcontents
\clearpage

\fancyhf{}
\fancyhead[L]{\thepage}

\chapter{A feladat leírása}
A feladat egy vagy több állatkert hálózatát mutatja be, amiben helyet kapnak az egyes állatkertekben dolgozók, azok feladatai, az állatok és élőhelyeik, eledelük, az eledelt gyártó cégek, illetve az állatok örökbefogadói, ha vannak. Mind az ER modell tervezésben és mind az XML megvalósításban angol nyelvet használtam, ugyanis ez a legelterjedtebb nyelv a programozásban\\
Összesen 6 egyedet hoztam létre, melyek a következők:
\begin{itemize}
	\item Employee,
	\item Site,
	\item Habitat,
	\item Animal,
	\item Food,
	\item User
\end{itemize}

\indent Legelőször is érdemes pár szót szólni a \textbf{Site} egyedről. Innen indul ki minden. Ez az egyed tárolja el az egyes
állatkertek legfőbb tulajdonságait, mint pl. név, terület vagy éppen nyitva tartás. Elsődleges kulcsa a \texttt{site\_id}, ami
az állatpark azonosítója.

A Site és az \textbf{Employee} egyed között egy \texttt{1:N} kapcsolat van, mivel egy állatkerthez több dolgozó is tartozhat,
de egy dolgozó, csak egy állatkerthez tartozhat. Az \texttt{1:N} kapcsolat neve: \textbf{Works}. Egy dolgozónak van azonosítója,
vezeték és keresztneve (ami ER modellben egy többágú tulajdonság), neme, születési dátuma és ami a legfontosabb, a dolgozó feladatai, posztjai, amiből lehet egy vagy több, így ez egy többértékű tulajdonság lesz. Ez azért fontos, mivel a relációs modellnél ez a tulajdonság egy külön táblát kap majd, amiben lesz a posztnak egy id-ja, a poszt neve, illetve, hogy kihez tartozik.

Egy állatkerthez több élőhely is tartoztat, de egy élőhely csak egy állatkerthez tartozik. Ezt ábrázolja a \textbf{Manage} kapcsolat, ami \texttt{1:N} kapcsolattal köti össze a Site és a \textbf{Habitat} egyedeket. Az élőhelynek nincsenek ,,extra'' tulajdonságai, van egy azonosítója, neve, térképen való elhelyezkedése, leírása és kapacitása, hogy mennyi állatot
képes egyszerre befogadni.

Az \textbf{Occupy} kapcsolat szintén \texttt{1:N} kapcsolattal köti össze a Habitat-ot az \textbf{Animal}-lel. Az állatnak van azonosítója, neve, faja és leírása.

Itt jön a legelső \texttt{N:M} kapcsolat, az \textbf{Eat}, aminek lesz tulajdonsága, a \texttt{feeding\_time}, az etetési idő. Fontos, hogy megjegyezzük, az \texttt{N:M} kapcsolat külön kapcsolótáblát fog kapni a relációs modellben. Az Eat köti össze az Animalt
a \textbf{Food}-dal, ami az állat eledelét modellező egyed. Ennek van azonosítója, neve, egy \texttt{boolean} (logikai) értéke, ami azt dönti el, hogy finom-e az adott eledel, vagy sem. Ezen kívül van egy többértékű tulajdonsága is, az eledeleket gyártó cégek, amik szintén külön táblát fognak majd kapni a relációs modellben.

Az állatokat örökbe is lehet fogani bizonyos \textbf{User}-eknek, ezt a \textbf{Favor} \texttt{1:1} kapcsolat modellezi. Talán a Usernek
van a legtöbb tulajdonsága ebben az adatbázisban. Van természetesen azonosítója, két neve (vezeték és keresztnév), neme, bejelentkezési adatai (felhasználónév, jelszó), mivel online szeretnénk lebonyolítani az állatok örökbefogadását. Ezen kívül címe is van a felhasználónak, ami az irányítószám, város, utca, házszám tulajdonságokból tevődik össze.

\chapter{I. feladat - XML/XSD létrehozás}

\section[ER modell]{A feladat ER modellje}

\begin{figure}[h]
	\centering
	\includegraphics[width=0.999\linewidth]{ERKLNSPG.png}
	\caption{A feladat ER modellje}
\end{figure}

\section[XDM modell]{A feladat XDM modellje}

\indent\indent A konvertáláskor figyelembe kell venni az ER modell során definiált kapcsolatokat, azok típusait (\texttt{1:1, 1:N, N:M}), illetve az entitások elsődleges kulcsait is. Minden \textit{egy-több} kapcsolat esetében ahhoz az elsődleges kulcshoz kerül a szaggatott nyíl, ahol az ER modellben a ,,több'' szerepel. Az ,,Eat'' és a ,,Favor'' kapcsolatokat kivéve, mindenhol \texttt{1:N} kapcsolat szerepel az ER modellben, így az XDM mindenhol majdnem hasonlóan fog kinézni. Az \texttt{N:M} kapcsolat esetében egy új modellt veszünk fel, tulajdonsággal és \textit{primary key}-el együtt természetesen, ahonnan a nyilakat a fő entitásokhoz húzzuk. A többágú tulajdonságok itt is több tulajdonsággal rendelkeznek, a többértékű tulajdonságok itt nem kapnak külön modellt. Az XDM modell gyökéreleme: \textbf{Zoo\_KLNSPG}

\begin{figure}[h]
	\centering
	\includegraphics[width=1.01\linewidth]{XDMKLNSPG.png}
	\caption{A feladat XDM modellje}
\end{figure}

A ,,Works'' kapcsolat \texttt{1:N}, ahol a több érték az \textit{Employee}-hoz kerül, így a kapcsolatot is az \textit{emp\_id}-hez húzzuk. Szintén ugyan ez a helyzet a ,,Manage'' kapcsolatnál is, ahol a rombuszt a \textit{habitat\_id}-hoz húzzuk. Az \textit{Animal} modell egy különleges, ide 3 kapcsolatot is húzunk, melyek:

\begin{itemize}
	\item ,,Manage'' a \textit{Habitat}-ból
	\item ,,AnimalEat'' az \textit{Eat}-ből és végül
	\item ,,Favor'' a \textit{User}-ből 
\end{itemize}
Kettő \texttt{1:N} kapcsolat és egy \texttt{1:1} kapcsolat húz ide.

\section[Az XML dokumentum]{Az XDM modell alapján XML dokumentum készítése}
\indent\indent Az \texttt{XMLKLNSPG.xml} dokumentumot \textit{Visual Studio Code}-ban hoztam létre, és \texttt{XML 1.0} szabvány szerint készült el. A dokumentumhoz hozzá kötöttem az \texttt{XMLSchemaKLNSPG.xsd} XSD file-t, és definiáltam az egyedeket az XML szabályainak megfelelően. Ahol szükséges volt, gyermek elemeket, valamint attribútumokat használtam a tagok azonosításához.

\begin{lstlisting}[caption={Az XML dokumentum}]
	<?xml version="1.0" encoding="UTF-8"?>
	
	<Zoo_KLNSPG xmlns:xsi="http://www.w3.org/2001/XMLSchema-instance" xsi:noNamespaceSchemaLocation="XMLSchemaKLNSPG.xsd">
	
	<!-- Employee peldanyok-->
	<Employee emp_id="1">
	<first_name>Kovacs</first_name>
	<last_name>Janos</last_name>
	<birth_date>1979-11-02</birth_date>
	<sex>M</sex>
	</Employee>
	<Employee emp_id="2">
	<first_name>Jakab</first_name>
	<last_name>Jozsef</last_name>
	<birth_date>1954-12-06</birth_date>
	<sex>M</sex>
	</Employee>
	<Employee emp_id="3">
	<first_name>Balogh</first_name>
	<last_name>Boglarka</last_name>
	<birth_date>2000-11-04</birth_date>
	<sex>F</sex>
	</Employee>
	
	<!-- Site peldanyok-->
	<Site site_id="1" Works="3" Manage="1">
	<name>Miskolc Allatkert</name>
	<area>212000</area>
	<opening_hours>09:00 - 17:00</opening_hours>
	</Site>
	<Site site_id="2" Works="1" Manage="2">
	<name>Fovarosi Allat-es Novenykert</name>
	<area>184001</area>
	<opening_hours>09:00 - 17:30</opening_hours>
	</Site>
	<Site site_id="3" Works="2" Manage="3">
	<name>Debreceni Allatkert es Vidampark</name>
	<area>170000</area>
	<opening_hours>09:00 - 15:30</opening_hours>
	</Site>
	
	<!-- Habitat peldanyok-->
	<Habitat habitat_id="1" Occupy="3">
	<name>Medve park</name>
	<location>#3</location>
	<description>Az allatkerti medvek elohelye. Jelenleg harom medve talalhato itt, Jazmin, Andor es Matyko. Szeretik a latogatokat, mindig erdeklodve nezelodnek.</description>
	</Habitat>
	<Habitat habitat_id="2" Occupy="1">
	<name>Muflonok dombja</name>
	<location>#22</location>
	<description>Vadasparkunk muflonjai itt talalhatoak. Baratsagosak, turista kedvelok, szeretik a finom falatokat.</description>
	</Habitat>
	<Habitat habitat_id="3" Occupy="2">
	<name>Szurikatak szigete</name>
	<location>#18</location>
	<description>Ki ne imadna a kis erdeklodo szurikatakat. Nalunk rogton 4-et is orokbe fogadhat, vagy csak latogathat is.</description>
	</Habitat>
	
	<!-- Animal peldanyok-->
	<Animal animal_id="1">
	<name>Matyko</name>
	<racial>Medve</racial>
	<description>Az allatkert egyik kan medveje</description>
	</Animal>
	<Animal animal_id="2">
	<name>Kis Hegyes</name>
	<racial>Muflon</racial>
	<description>Az allatkert nosteny muflona</description>
	</Animal>
	<Animal animal_id="3">
	<name>Mokas</name>
	<racial>Szurikata</racial>
	<description>Az allatkert legfiatalabb szurikataja</description>
	</Animal>
	
	<!-- Food peldanyok-->
	<Food food_id="1">
	<name>Fagyasztott nyershus</name>
	<is_delicious>false</is_delicious>
	<company>Family Frost</company>
	</Food>
	<Food food_id="2">
	<name>Sargarepa</name>
	<is_delicious>true</is_delicious>
	<company>Magyar Zoldseg</company>
	</Food>
	<Food food_id="3">
	<name>Sult husi</name>
	<is_delicious>true</is_delicious>
	<company>Family Frost</company>
	</Food>
	
	<!-- Eat peldanyok-->
	<Eat eat_id="1" FoodEat="1" AnimalEat="3">
	<feeding_time>09:00:00 18:00:00</feeding_time>
	</Eat>
	<Eat eat_id="2" FoodEat="3" AnimalEat="1">
	<feeding_time>07:00:00 20:00:00</feeding_time>
	</Eat>
	<Eat eat_id="3" FoodEat="2" AnimalEat="2">
	<feeding_time>06:00:00 14:00:00 20:00:00</feeding_time>
	</Eat>    
	
	<!-- User peldanyok-->
	<User user_id="1" Favor="3">
	<username>Allatbarat</username>
	<password>allat123</password>
	<sex>M</sex>
	<first_name>Kiss</first_name>
	<last_name>Sandor</last_name>
	<post_code>8200</post_code>
	<city>Veszprem</city>
	<street>Petofi Sandor utca</street>
	<number>3</number>
	</User>
	<User user_id="2" Favor="1">
	<username>Vadoc</username>
	<password>fegyo02</password>
	<sex>M</sex>
	<first_name>Fegyver</first_name>
	<last_name>Sandor</last_name>
	<post_code>4024</post_code>
	<city>Debrecen</city>
	<street>Kossuth utca</street>
	<number>26</number>
	</User>
	<User user_id="3" Favor="2">
	<username>Possumluvr</username>
	<password>possumlover</password>
	<sex>F</sex>
	<first_name>Kazai</first_name>
	<last_name>Eszter</last_name>
	<post_code>3521</post_code>
	<city>Miskolc</city>
	<street>Uj elet utca</street>
	<number>24</number>
	</User>
	
	</Zoo_KLNSPG>
\end{lstlisting}
\clearpage

\section{Az XML dokumentum alapján XMLSchema készítése}
\indent\indent Az \texttt{XMLSchemaKLNSPG.xsd} séma file leírja mindazon megkötéseket, amelyeknek az XML dokumentumnak meg kell felelnie. Itt definiálunk minden típust, amit az XML file-ban használni szeretnénk, valamint az adatbázis kapcsolatait \texttt{xs:unique} és \texttt{xs:keyref} bejegyzésekkel hozom létre.

\begin{lstlisting}[caption={Az XSD dokumentum}]
	<?xml version="1.0" encoding="UTF-8"?>
	<xs:schema xmlns:xs="http://www.w3.org/2001/XMLSchema">
	
	<!-- Sajat egyszeru tipusok definialasa -->
	
	<!-- Altalanos sajat tipusok-->
	<xs:element name="name" type="xs:string"/>
	<xs:element name="first_name" type="xs:string"/>
	<xs:element name="last_name" type="xs:string"/>
	<xs:element name="sex" type="sexType"/>
	<xs:element name="description" type="xs:string"/>
	
	<!-- Employee sajat tipus-->
	<xs:element name="birth_date" type="dateType"/>
	
	<!-- Site sajat tipusok-->
	<xs:element name="area" type="xs:integer"/>
	<xs:element name="opening_hours" type="xs:string"/>
	
	<!-- Habitat sajat tipus-->
	<xs:element name="location" type="xs:string"/>
	
	<!-- Animal sajat tipus-->
	<xs:element name="racial" type="xs:string"/>
	
	<!-- Food sajat tipus-->
	<xs:element name="is_delicious" type="xs:boolean"/>
	<xs:element name="company" type="xs:string"/>
	
	<!-- Eat sajat tipus-->
	<xs:element name="feeding_time" type="timeListType"/>
	
	<!-- User sajat tipusok-->
	<xs:element name="username" type="xs:string"/>
	<xs:element name="password" type="xs:string"/>
	<xs:element name="post_code" type="xs:string"/>
	<xs:element name="city" type="xs:string"/>
	<xs:element name="street" type="xs:string"/>
	<xs:element name="number" type="xs:string"/>
	
	<!-- Simple types-->
	<xs:simpleType name="dateType">
	<xs:restriction base="xs:date">
	<xs:minInclusive value="1940-01-01"/>
	<xs:maxInclusive value="2000-12-31"/>
	</xs:restriction>
	</xs:simpleType>
	
	<xs:simpleType name="sexType">
	<xs:restriction base="xs:string">
	<xs:enumeration value="M"/>
	<xs:enumeration value="F"/>
	</xs:restriction>
	</xs:simpleType>
	
	<xs:simpleType name="timeListType">
	<xs:list itemType="xs:time"/>
	</xs:simpleType>
	
	<!-- Complex types-->
	<xs:complexType name="employeeType">
	<xs:sequence>
	<xs:element ref="first_name"/>
	<xs:element ref="last_name"/>
	<xs:element ref="birth_date"/>
	<xs:element ref="sex"/>
	</xs:sequence>
	<xs:attribute name="emp_id" type="xs:integer" use="required"/>
	</xs:complexType>
	
	<xs:complexType name="siteType">
	<xs:sequence>
	<xs:element ref="name"/>
	<xs:element ref="area"/>
	<xs:element ref="opening_hours"/>
	</xs:sequence>
	<xs:attribute name="site_id" type="xs:integer" use="required"/>
	<xs:attribute name="Works" type="xs:integer" use="required"/>
	<xs:attribute name="Manage" type="xs:integer" use="required"/>
	</xs:complexType>
	
	<xs:complexType name="habitatType">
	<xs:sequence>
	<xs:element ref="name"/>
	<xs:element ref="location"/>
	<xs:element ref="description"/>
	</xs:sequence>
	<xs:attribute name="habitat_id" type="xs:integer" use="required"/>
	<xs:attribute name="Occupy" type="xs:integer" use="required"/>
	</xs:complexType>
	
	<xs:complexType name="animalType">
	<xs:sequence>
	<xs:element ref="name"/>
	<xs:element ref="racial"/>
	<xs:element ref="description"/>
	</xs:sequence>
	<xs:attribute name="animal_id" type="xs:integer" use="required"/>
	</xs:complexType>
	
	<xs:complexType name="foodType">
	<xs:sequence>
	<xs:element ref="name"/>
	<xs:element ref="is_delicious"/>
	<xs:element ref="company"/>
	</xs:sequence>
	<xs:attribute name="food_id" type="xs:integer" use="required"/>
	</xs:complexType>
	
	<xs:complexType name="eatType">
	<xs:sequence>
	<xs:element ref="feeding_time"/>
	</xs:sequence>
	<xs:attribute name="eat_id" type="xs:integer" use="required"/>
	<xs:attribute name="FoodEat" type="xs:integer" use="required"/>
	<xs:attribute name="AnimalEat" type="xs:integer" use="required"/>
	</xs:complexType>
	
	<xs:complexType name="userType">
	<xs:sequence>
	<xs:element ref="username"/>
	<xs:element ref="password"/>
	<xs:element ref="sex"/>
	<xs:element ref="first_name"/>
	<xs:element ref="last_name"/>
	<xs:element ref="post_code"/>
	<xs:element ref="city"/>
	<xs:element ref="street"/>
	<xs:element ref="number"/>
	</xs:sequence>
	<xs:attribute name="user_id" type="xs:integer" use="required"/>
	<xs:attribute name="Favor" type="xs:integer" use="required"/>
	</xs:complexType>
	
	<!-- A gyokerelem osszetett tipusa -->
	<xs:complexType name="zooType">
	<xs:sequence>
	<xs:element name="Employee" type="employeeType" minOccurs="3" maxOccurs="unbounded"/>
	<xs:element name="Site" type="siteType" minOccurs="3" maxOccurs="unbounded"/>
	<xs:element name="Habitat" type="habitatType" minOccurs="3" maxOccurs="unbounded"/>
	<xs:element name="Animal" type="animalType" minOccurs="3" maxOccurs="unbounded"/>
	<xs:element name="Food" type="foodType" minOccurs="3" maxOccurs="unbounded"/>
	<xs:element name="Eat" type="eatType" minOccurs="3" maxOccurs="unbounded"/>
	<xs:element name="User" type="userType" minOccurs="3" maxOccurs="unbounded"/>
	</xs:sequence>
	</xs:complexType>
	
	<!-- A gyokerelem definicioja -->
	<xs:element name="Zoo_KLNSPG" type="zooType">
	
	<!-- Elsodleges kulcsok -->
	<xs:key name="EmployeeKey">
	<xs:selector xpath="Employee"/>
	<xs:field xpath="@emp_id"/>
	</xs:key>
	
	<xs:key name="SiteKey">
	<xs:selector xpath="Site"/>
	<xs:field xpath="@site_id"/>
	</xs:key>
	
	<xs:key name="HabitatKey">
	<xs:selector xpath="Habitat"/>
	<xs:field xpath="@habitat_id"/>
	</xs:key>
	
	<xs:key name="AnimalKey">
	<xs:selector xpath="Animal"/>
	<xs:field xpath="@animal_id"/>
	</xs:key>
	
	<xs:key name="FoodKey">
	<xs:selector xpath="Food"/>
	<xs:field xpath="@food_id"/>
	</xs:key>
	
	<xs:key name="EatKey">
	<xs:selector xpath="Eat"/>
	<xs:field xpath="@eat_id"/>
	</xs:key>
	
	<xs:key name="UserKey">
	<xs:selector xpath="User"/>
	<xs:field xpath="@user_id"/>
	</xs:key>
	
	<!-- Idegen kulcsok -->
	<xs:keyref name="SiteWork" refer="EmployeeKey">
	<xs:selector xpath="Site"/>
	<xs:field xpath="@Works"/>
	</xs:keyref>
	
	<xs:keyref name="SiteManage" refer="HabitatKey">
	<xs:selector xpath="Site"/>
	<xs:field xpath="@Manage"/>
	</xs:keyref>
	
	<xs:keyref name="HabitatOccupy" refer="AnimalKey">
	<xs:selector xpath="Habitat"/>
	<xs:field xpath="@Occupy"/>
	</xs:keyref>
	
	<xs:keyref name="EatFood" refer="FoodKey">
	<xs:selector xpath="Eat"/>
	<xs:field xpath="@FoodEat"/>
	</xs:keyref>
	
	<xs:keyref name="EatAnimal" refer="AnimalKey">
	<xs:selector xpath="Eat"/>
	<xs:field xpath="@AnimalEat"/>
	</xs:keyref>
	
	<xs:keyref name="UserAnimal" refer="AnimalKey">
	<xs:selector xpath="User"/>
	<xs:field xpath="@Favor"/>
	</xs:keyref>
	
	<!-- Az 1:1 kapcsolat-->
	<xs:unique name="UserAnimalConnect">
	<xs:selector xpath="UserKey"/>
	<xs:field xpath="@Favor"/>
	</xs:unique>
	
	</xs:element>
	
	</xs:schema>
\end{lstlisting}
\chapter{II. feladat - DOM}
\section{Adatolvasás}
\section{Adatmódosítás}
\section{Adatlekérdezés}
\section{Adatírás}

\end{document}